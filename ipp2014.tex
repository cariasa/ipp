\documentclass[a4paper,10pt]{article}
\usepackage[utf8]{inputenc}
\usepackage{authblk}
\usepackage[margin=2cm]{geometry}
\renewcommand{\baselinestretch}{1.5}
\usepackage[round]{natbib}
\usepackage{hyperref}


%opening
\title{Population as Auditor of an Election Process in Honduras: VotoSocial}
\author[1,2]{Carlos Roberto Arias (cariasa@unitec.edu)}
\author[1,3]{Jorge Antonio Garcia (jorge@icoms.co)}
\author[3]{Alejandro Corpeño (corp@icoms.co)}


\affil[1]{Facultad de Ingenier\'{i}a, UNITEC, Tegucigalpa, Honduras}
\affil[2]{Instituto de Investigaci\'{o}n de Pol\'{i}ticas P\'{u}blicas, UNITEC, Tegucigalpa, Honduras}
\affil[3]{Icoms Technologies S de RL, Tegucigalpa, Honduras}

\begin{document}

\maketitle

\begin{abstract}

\end{abstract}

\section{Introduction}
It is accepted among Hondurans that the government is corrupt, this is supported by the evidence found by Transparency International that rates Honduras with a score of 26, with only 37 other countries considered to be more corrupt than Honduras \citep{transp}. This fact is affecting the Honduran society in many dimensions, one of them is the impact it has on the education and thus the democratic maturity of the general population. With a 14.5\% of the population above 15 years of age that cannot read or write \citep{bchrep}, and with 64.5\% of the population living under the line of poverty \citep{wbdata}, this country's educated people is a minority.


However a minority can have a powerful impact in the country, when even a small group can shake the status quo of a government \citep{saadia2014} as it has happened in the recent government elections held in October 2013. During this process a Crowdsourcing System was built called VotoSocial: \url{http://votosocial.org}, where 97\% of the presidential polling records where reviewed by the people participating with the system.

In this electoral process eight political parties participated for the President Position, one of the candidates being the former President of Congress. This fact and the previous turmoil produced by the events of 2009 where former President Manuel “Mel” Zelaya was deposed, had the country in a state of tension. Even with the political stress, this past elections proved to be representative as more than 60\% of the registered voters participated with their votes. In addition to the high turnout, in this elections the official polling tables records were digitalized and made public by the Supreme Elections Tribunal (TSE) through their elections site SIEDE: \url{http://siede.tse.hn}.

During the process there was a perception that the elections were fraudulent, specifically the process of digital input of the records to the TSE computer system. Most of this perception was due to the political power that one of the candidates had at the moment of the elections, as pictures of inconsistent polling records favoring this candidate appeared in social media, and partly because of the awareness of the general public of the corruption of the government. The lack of confidence in the elections system of Honduras can be seen in the infamous statement by former president Zelaya where he stated that it is necessary to have a 10\% fraud to be able to beat the system \citep{melvid}, additionally studies have shown that only 23\% of Hondurans believe in the electoral process \citep{romero2014}.

Considering all these facts, some people started to organize procedures to check the official records, and to find a way to report this to the general public. Some used Facebook as their propagation method, and Google Docs as a way to register the potential anomalies found in the counting process of the official polling records. Some other people, among the authors of this paper, decided to take this a step further, and developed the \textit{VotoSocial} platform, to allow people to verify the government counting of the official polling records.

Democracy is not an easy feat, especially in a developing country as Honduras, but Crowdsourcing systems like \textit{VotoSocial}, can help a minority of a population have a deep impact in the society, allowing some of the more educated people help discover anomalies in the government proceedings, making this information public and explaining how this affects the general public. Furthermore, this initiative allowed users that would just complain about the government in Facebook, to use their energy in a positive constructive way, and to involve them in the political life of their country, something that they would not have been able to do without \textit{VotoSocial}.

These kinds of systems will help society get a better participation in the governance of a country, because they will make people be heard by their government and will increase general public awareness. In addition to these benefits, \textit{VotoSocial} provides a path to a more transparent electoral process, so much that it has serve as an inspiration to other Crowdsourcing systems like \url{http://contemosnosotros.org/} in the neighbor country of El Salvador. 

Following this introduction a brief background description of crowdsourcing and the political history of Honduras is presented, after this a panoramic view of the elections in 2013 is shown, followed by a statistical analysis that the \textit{VotoSocial} platform allowed to do. Finally future and potential work is described and conclusions are given.


\section{Background}

\subsection{Crowdsourcing}

Cyberactivism is reshaping the way that policy has traditionally been conducted, whose perspectives are included and how awareness is risen among the people \citep{milan2013}, cyberactivism activitites include among others crowdsourcing. Crowdsourcing is a process where an online community collaborates to reach a goal, where each participant contributes with a small portion, and these contributions are later integrated to provide a solution to a bigger problem. In a manner, it builds a big solving machine using humans as computational components, thus enlisting people to solve a variety of problems \citep{doan2011}.

According to \cite{doan2011} crowdsourcing operations need to consider the amount of manual effort, the role of the human users and whether the system is going to be standalone or piggy-back on another system. The amount of manual effort refers to how much time and effort each human user is expected to give to the system, most systems expect very little from users, and work with a considerable number of users. Additionally, the function of these users needs to be clearly stablished, in such a way that it is difficult for users to introduce noise to the solution of the problem at hand. Lastly, some systems rely on the existence of another system, creating a symbiotic relationship, for instance the reCAPTCHA (\url{https://www.google.com/recaptcha/intro/index.html}) system that piggy-backs on other systems to help digitize text, annotate images and help in the building of machine learning datasets; some other systems are standalone such like the SETI@Home (\url{http://setiathome.berkeley.edu/}). In addition to these considerations, there are several dimensions of crowdsourcing that establish how to recruit and retain the users, what each user can do, how to combine their contributions and how to evaluate them. For instance in one crowdsourcing solution for disaster relief \citep{gao2011}, users would be recruited by letting a wide range of the population know the availability of the system, in this instance people use Twitter to send information messages. The retention of users is achieved by the good will of people to collaborate with the emergency relief efforts. Users only needed to send their tweets with a identifiable hash tag, and then the system would combine these tweets by mining the tweets. This particular system had the challenge of data validation, but it serves as a good example on how this dimensions need to be dealt with.

Crowdsourcing systems do not exist without problems and challenges, among them there is the need to evaluate the correctness and validity of the user contributions, how to reach enough users to have a significative contribution to the system and how to make people accountable for their participation. Each crowdsourcing system faces these challenges their own way, depending on their specific goals and domain.

Needless to say that thanks to social media, crowdsourcing offers a powerful tool to collect information and users can also be used to validate the data \citep{gao2011}. Succesful stories on how crowdsourcing has been used can be found around the globe, like its usage in disaster relief \citep{yin2012, gao2011} and its use in the electoral process of Honduras and El Salvador in 2013.

For crowdsourcing systems to be successful they need to keep its human users motivated, this can be achieved by convincing them of the necessity of the service, instant gratification, fame and reputation \citep{doan2011}. The last two factors can be reached by attaching a social media component to the crowdsourcing system, so that users can show off their accomplishments with their peers. The use of social media also helps with the challenge of reaching more users, in addition to become a source of information and communication \citep{yin2012}, and helps political bodies as a mean to gather support, movilize people and to get political messages spread \citep{map2014}.

By involving social media, many of the crowdsourcing challenges are faced and solved, and in addition the systems get a broader audience, so that results can also be shared with more people, providing better situation awareness, new paths of communication and opportunities for assistance \citep{gao2011}. However the crowdsourcing systems need to provide appropriate and rapid response and results to keep active and collaborating users, and passive and read-only users engaged with the system.


\subsection{Honduras Political History}

Honduras political history is marked with bipartidism between the Liberal Party and the National Party, and the frequent military government in between. The longest, and most remembered, dictactorship was by General Tiburcio Carías in 1932, and when it finished in 1948 the country started its tendency to democracy.

Following the dictactorship that lasted from 1932 to 1948, the General Tiburcio Carias called for elections. However the results of this process were already arranged in such way that Juan Manuel Galvez, from the National Party, won this elections with 99.8\% of the votes in favor. Later, in 1954 presidential elections were held again, the opposition party, the Liberal Party won the elections, but since the law required an absolute mayority, the Vice President took over the government. In 1956 the opposition of the ruling government was persecuted, a candidate was exiled and state resources were movilized to enable clientelism. By means of fraud the government party achieved 89.4\% of the votes and full representation in Congress \citep{romero2014}. This election was followed by a coup d'\'{e}tat, and after that a Constituent Assembly was brought, later called for elections where Ramon Villeda Morales, a Liberal Party militant, was elected President, he was later deposed by another coup from the military. This military government called again to elections, and the elected President Ramon Ernesto Cruz, was also deposed by the military a little more than a year later of being elected. During the military government a new Constituent Assembly was called, a new Constitution was writen, and elections were called again in 1982.

From 1948 to 1980 coup d'\'{e}tat, fixed elections and fraud where interleaved until in the early 1980's a new National Constitution was writen and an electoral process began again in 1982. Democracy was then restored, and has continued periodically every four years, without interruption of the military raising to power. The only exception to this democratic continuity was the deposition of President Manuel Zelaya, from the Liberal Party, in 2009. A factor affecting the political turmoil from 1948 to 1980 was that the scarce available media was unable to work well due to the restrictions of freedom of speech. 

During this democratic streak, fraud has not been as crude as it was 30 years ago \citep{romero2014}, and Honduras has been in the vanguard of electoral process in comparison to its immediate neighbors, for instance the advances in 1997: stablished one ballot, domiciliary voting, use of national ID card as voter registration \citep{romero2014}.

The events around the 2009 political crisis, when elected President Manuel Zelaya was deposed by Congress, and exiled by the Military, confirmed the fragile capacity of public action of Honduran society \citep{romero2014}. The following 2009 elections in Honduras had one of the lowest turnout in recent history having the following distribution of votes: 17.06\% independents, 56.05\% loyal militants, 45.97\% electoral participation, 18.47\% victory margin. According to \cite{gonza2014} during this elections 2009 22\% of the voters participated in one form or another of vote buying. Vote buying strategy was mostly oriented to loyal militants, in the form of targeted turnout and movilization \citep{gonza2014}. Porfirio Lobo won this elections ending the pattern from 1982 of two liberal party governments preceeding a national party government.

Survey results show that even though there is a periodicity to the democratic process, the public still does not trust the general process and transparency of the elections \citep{romero2014}, other research shows that 48.3\% believe that democracy is worst now, only 5\% is satisfied with the current democracy \citep{latinbar}. Even though Honduras is improving technically in its electoral process, there is a clear lack of confidence in the results, this is mostly due to the historical tendency to fraud and vote buying.

\section{Government Elections 2013: The Birth of VotoSocial}

Elections were held in November 2013, eight political parties participated with a total turnout of 61.16\%, a significative increase from the previous elections. This elections were historically unique since they ended the usual bipartidism that is usual around the world, in this elections the Liberal Party, one of the traditionally more powerful parties, only won in one department of the republic. Another particularity of this elections was that new political parties won whole departments, something that never happened before with any of the other smaller parties. A new party called PAC won one department, four departments were taken by LIBRE a new party derived from the Liberal Party after the crisis of 2009, and the rest of the country was won by the National Party.

The PAC party presented for the first time a different proposal, filling a vacuum that traditional parties have not been able to fill. People are expecting a top-down approach where politicians use more social media, reach the population and listen to the people using this technology \citep{map2014}. This was reflected in the elections counts where PAC achieved good results in urban areas where there is greater development, the downside was that there was little or no influence in the rural areas. An additional consequence to the usage of social media, is that this particular party attracted younger sectors that usually did not participate in the political processes \citep{romero2014}.

The National Party, represented by Juan Orlando Hernández at that time President of National Congress, mostly won in the western and southern parts of the country, in these regions there is a greater rural population, a population with less development that wanted to preserve the ``Bono 10,000'' program. During the 2010 - 14 term of President Porfirio Lobo, the``Bono 10,000'' program was created. ``Bono 10,000'' is a government program with the goal of transfering money to homes categorized as in a state of extreme poverty, with children properly enrolled in the public school system and that regularly assist to school. It is part of the Millenium Development Goals \citep{mdg2006} to eradicate extreme poverty.To reach this goal the ``Bono 10,000'' program was created as a Presidential Program for Health, Education and Nutrition \citep{bono10k}. This shows consistency with the social policy in Latinamerica that is characterized by ``conditioned subsidies'' \citep{romero2014}. This program started by former President Porfirio Lobo is continued by current President Juan Orlando Hern\'{a}ndez. 

During the 2013 elections middle and upper class had higher turnout rates, urban lower class contrasted with rural people that are more participative, political parties networks worked very well in rural parts of the country \citep{romero2014}. This behavior as expected shows how low income citizens usually respond positively to immediate benefit \citep{kit2000}, a conduct seen in developing countries, which relates to the correlation between gross domestic product and vote buying. Evidence corroborates that social benefits for lower income classes, makes these people more susceptible to vote buying, thus showing that the rational theories are held in Honduras \citep{gonza2014}. Besides income, the other factor affecting the vote is the intensity of political sympathy, that may result in the movilization of followers. This form of vote buying takes the shape of electoral participation buying. This is consistent to the voters behavior in the United States where socioeconomic status correlates to geographical location of voters and also to political preferences of voters \citep{osborn2010}.

Another characteristic of the 2013 elections was the participation of nine political parties. Having so many political parties opened the door to polling tables credentials trafic, hence helping bigger political parties gain control. This was seen during the 2013 elections and documented in a TV interview where a representative of the National Party was holding a credential of another Party \citep{vidap}. The polling table remains the weak link in the electoral process as this is the least supervised spot \citep{romero2014}. Bigger and traditional parties usually possess resources and have a greater capacity to turn on the vote buying machinery. To achieve this, it requires knowledge of the local population, budget, trust networks, judicial protection \citep{gonza2014}. Judicial protection was achieved by current president, at the time that he was president of Congress by dismissing four Supreme Court Justices, specifically the ones in charge of declaring violations to the National Constitution \citep{csj2012}.

In the midst of the official vote counting by the Supreme Electoral Tribunal (TSE), VotoSocial is born. VotoSocial team grabbed all the digitalized polling records. The process of retrieving these records was done in batch, downloading the records as these were being made available by the TSE. VotoSocial then allowed the system's users to see the scanned record, and to register the values, digitalizing this way the results of each of 15,637 polling tables, this represents 97\% of all the polling stations in the country, the remaining 3\% was not verified as it was never made public by the TSE. Users were also in charge of the validation of this process, so every time a transcription user digitized a record, three different users had to validate that this transcription was correct. There were 6,232 unique visits to the site, 1,673 people registered in VotoSocial, 710 users were actively transcribing records, and 879 people participated reviewing the transcriptions. The whole process took six days, however it was done altogether in less than 48 hours. In the first day programming and testing was done and 1\% of the records were processed, during the following two days 88\% of the records were processed, hours after these were available for the VotoSocial users. The rest of the records were processed during few hours in the last three days.

According to \cite{doan2011} VotoSocial architecture is explicit and assigns the users with task execution, all effort is done online, and unlike other systems that send tickets when a user finds a mistranscription \citep{haaf2013}. In an effort to find correct results a method of triple verification was used, this allowed for a user controled validation of the data. Users were recruited using social networks as Facebook and Google+, basically taking advantage of the word of mouth. Data was integrated by simple addition, so integration was not actually a challange of the platform. Opposed to other crowdsourcing systems, VotoSocial did not form a community as such, since there was not explicit communication between the users inside the platform. Decisions were always made agilely by the system operators taking user comments into account, much the way a software is developed nowadays. The operations that users were allowed to do were as digitizers and reviewing. Digitizing consisted in transcribing the official record, and reviewing consisted in the revision of this transcription. For a record transcription to be accepted three different persons had to review and check that the record has been correctly transcribed.






In May 21st the authors requested oficial data to the Public Information Access Institute \url{http://www.iaip.gob.hn/} however after three months there was no response on their behalf.





Higher turnout usually means a greate demand of the public for results \citep{mac2003}, this is not true for Honduras where abstention and comfort are not aligned, since even when there is high turnout people do not consistently demand results from the government \citep{romero2014}.

-------------------------



VotoSocial did not find significant discrepancies between the official results of the TSE and its own results, thus showing that at least in the digitalization process in the TSE there were no “foul play”. Notwithstanding this, further analysis of the data showed a correlation between the percentage of voter turnout in the polling tables and the percentage of the voters of the winning candidate, where there should be none, this suggests that during the elections the winning party had an active encouragement to get more sympathizing people to the polls.

In the 1948 elections there was an electoral scaffolding that far from helping a fair election used a mechanism that guaranteed the victory of the ruling party. These mechanisms had the favor of the government resources, much like what happened in 2013, that even with the technical advances, the elections lacked the confidence of the general public \citep{romero2014}. Considering the population resentment and suspicions with this latest government, social media may gather together people, but crowdsourcing platforms may be needed to provide structure and systematization with the goal of reaching some positive change.


\section{Further Analysis}

\section{Future and Potential Work}

\section{Conclusions}


\newpage


\bibliographystyle{apalike}
\bibliography{./ipp}

\end{document}
