Processed
Crowdsourcing is defined as the enlistment of humans to solve a variety of problmes \citep{doan2011}
Crowdsourcing considerations  are degree of manual effort, role of human users, and whether the system is going to be standalone or piggy-back \citep{doan2011}
Crowdsourcing dimensions are how to recruit, how to retain, what a user can do, how to combine the contributions, how to evaluate \citep{doan2011}
Crowdsourcing enlists many users, helps solve a problem, problem is defined by system owners \citep{doan2011}
Challenges of crowdsourcing how to recruit users, how to evaluate users \citep{doan2011}
Crowdsourcing based on social media offers a powerful tool to collect information \citep{gao2011}
Crowdsourcing helps to validate data \citep{gao2011}
Crowdsourcing success has been recorded in disaster relief \citep{yin2012}, 
To retain collaborators in crowdsourcing there are several strategies necessary service, instant gratification, fame and reputation \citep{doan2011}
Social media has emerged as a popular space that rapidly provides sources of information and communication \citep{yin2012}
Online social media helps political bodies as a mean to gather support, movilize people and to get political messages spread \citep{map2014}
Crowdsourcing provides situation awareness, new paths of communication and opportunities of assistance. Crowdsourcing needs sense making, security and coordination \citep{gao2011}
Information for disaster relief needs to to be analyzed properly and rapidly to enhance the level of perception of the situation awareness \citep{yin2012} this will help to keep people motivated.



Non Processed

Online access is not a guarantee of reaching all the public if the State actively censors and controls online chanels \citep{map2014}, current low penetration of Internet makes online activities to pass under the radar of power actors.

When social media information is not visible to the government, this usually means that it is not being monitored by the government \citep{yin2012}

Government use of social media seeks two goals:  political popularity, and power and control through censorship \citep{saadia2014}

Social Networks use encouraged sectors to vote in Kenya \citep{map2014}

The correlation between voter turnout and voter choice is the product of two factors: voter movilization and voter defection \citep{osborn2010}

Political parties actions such as phone calls and email can encourage people to vote, having a direct effect on the turnout, this may benefit a political party depending on the specific characteristics of the country \citep{osborn2010}

To avoid abstentionism transportation to the polling table is a common practice \citep{romero2014}
Honduras' census lacks an efficient mechanism to register dead people, in the last elections there were 300,000 dead people registered to vote \citep{romero2014}